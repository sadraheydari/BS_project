\فصل{روش‌شناسی}

\قسمت{روند کلی}
برای دسته‌بندی خانوارها به کمک روش‌های مبتنی بر هوش مصنوعی، لازم است هر خانوار را در فضای برداری نمایش دهیم. استفاده مستقیم از ویژگی‌های موجود در دیتاست رفاه ایده مناسبی نیست چرا که اهمیت هر ویژگی با دیگری در تقسیم‌بندی ما تفاوت دارد.

برای حل این موضوع ابتدا ویژگی‌ها را به چند دسته تقسیم می‌کنیم. سپس برای هر دسته معیار شباهتی مشخص می‌کنیم. این معیار شباهت باید به گونه‌ای باشد که با دریافت ویژگی‌های مشخص از دو خانوار، میزان شباهت آن‌ها را  در مقیاس ۰ (کاملاً متفاوت) تا ۱ (کاملاً یکسان) خروجی دهد. سپس به‌ازای هر دسته از ویژگی‌ها یک ماتریس شباهت تولیدکرده که درایه 
$ij$
آن نشان‌دهنده میزان شباهت خانوار 
$i$ام
و
$j$ام
خواهد بود. بدیهی‌است که داده‌های قطری در این ماتریس همگی برابر با ۱ می‌باشند.

در ادامه با استفاده از الگوریتم 
t-SNE
ماتریس‌های ساخته‌شده را به فضای برداری با بعد دلخواه 
$n$
نگاشت می‌کنیم به صورتی که فاصله اقلیدسی خانوارهایی که در ماتریس شباهت زیادی باهم داشته‌اند کم بوده و فاصله خانوارهای بی‌شباهت زیاد باشد.
این عمل را برای همه دسته‌ها انجام داده و با به‌هم چسباندین نتایج حاصل هر خانوار را در فضای برداری با بعد دلخواه نگاشت کرده‌ایم.

بعد از آن تلاش می‌کنیم با استفاده از الگوریتم‌های مختلف دسته‌بندی،‌ خانوارها را در فضای برداری ساخته‌شده دسته‌بندی کرده و تا حد امکان آن‌ها را از یکدیگر تفکیک کنیم.

باید توجه کرد که به واسطه استفاده از الگوریتم 
t-SNE
نتایج حاصل مقداری تصادفی بوده ولی در هدف و نتیجه کار ما تأثیری ایجاد نمی‌کند.