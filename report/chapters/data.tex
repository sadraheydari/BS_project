\فصل{دادگان}

%برای این پروژه از داده‌های عمومی پایگاه رفاه ایرانیان استفاده می‌کنیم.


\قسمت{معرفی داده}

در این پروژه از داده‌های عمومی «پایگاه اطلاعات رفاه ایرانیان» استفاده می‌کنیم. این پایگاه اطلاعات نتیجه تلاش وزارت تعاون، کار و رفاه اجتماعی جمهوری اسلامی ایران برای ایجاد شناسنامه رفاهی-اقتصادی افراد است.

پایگاه اطلاعات رفاه ایرانیان، از تجمیع ۵۰ منبع داده‌ای تشکیل شده‌­است. در فاز نخست، ۲۵ منبع داده‌ای به طور کامل تجمیع و ساختار اصلی پایگاه اطلاعات رفاه ایرانیان (با بیش از ۶۰ جدول داده‌ای) بر اساس آن‌ها شکل گرفته‌است. بیش از ۳ میلیارد رکورد داده‌ای در پایگاه ­داده نهایی ذخیره و تجمیع شده‌ است. در حال حاضر، ۲۲۱ فیلد اطلاعاتی شناسایی شده‌اند که ابعاد مختلف هویتی شهروندان را به صورت مستقیم و غیر‌مستقیم توصیف می‌نمایند. بخشی از خروجی‌های پایگاه اطلاعات رفاه ایرانیان به صورت وب‌سرویس در اختیار ۲۰ سازمان مختلف قرار گرفته‌ ­است. علاوه‌بر‌آن، خدمات تحلیلی نیز به تعدادی از سازمان‌ها ارائه شده‌ است. این خدمات شامل استحقاق‌سنجی، بررسی همپوشانی، داده‌کاوی، وسع‌سنجی و ... می‌شود. (لینک به صفحه)

پایگاه ملی اطلاعات رفاه ایرانیان حاوی اطلاعات مفیدی در مورد ابعاد مختلف اجتماعی-اقتصادی  شهروندان ایرانی است که می‌تواند برای اهداف مختلف مورد تحلیل و بررسی اندیشمندان، جامعه‌شناسان، اقتصاددانان و به طور کلی جامعه محققان کشور قرار گیرد. از سمتی دیگر، این پایگاه شامل اطلاعات محرمانه و خصوصی شهروندان است که نقض این حریم خصوصی، ممکن است تبعات جبران ناپذیری در زمینه اعتماد عمومی ایجاد نماید. معاونت رفاه اجتماعی به منظور ایجاد یک مصالحه میان 1- امکان دسترسی جامعه محققان به داده‌های پایگاه اطلاعات ایرانیان جهت انجام مطالعات تحقیقاتی و 2- حفظ حداکثری حریم خصوصی شهروندان ایرانی، اقدام به تهیه نمونه‌های 2 درصدی از اطلاعات موجود در پایگاه ملی اطلاعات رفاه ایرانیان نموده است. در این نمونه‌برداری‌ها سعی شده است که نکات زیر به صورت جدی مدنظر قرار گیرند.


\قسمت{معرفی ویژگی‌ها}

دیتاست موجود شامل اطلاعات پانصدهزار خانوار و ۱٫۴۹۰٫۹۹۱ فرد ایرانی است و برای هر فرد ۴۸ ویژگی دارد. ویژگی‌های موجود در دیتاست را می‌توان به دسته‌های زیر تقسیم‌بندی کرد.

\شروع{فقرات}
\فقره{\textbf{اطلاعات شخصی}}
\begin{multicols}{2}
	\شروع{فقرات}
	\فقره شناسه فرد
	\فقره شناسه سرپرست خانوار
	\فقره تاریخ تولد
	\فقره جنسیت
	\پایان{فقرات}
	
\end{multicols}

\فقره \textbf{محل زندگی}
\begin{multicols}{2}
	\شروع{فقرات}
	\فقره کد پستی
	\فقره استان محل زندگی
	\فقره شهرستان محل زندگی
	\فقره شهری یا روستایی بودن
	\پایان{فقرات}
	
\end{multicols}

\فقره \textbf{اطلاعات حساب} (برای سال‌های ۹۵ تا ۹۸ و به تفکیک سال)
\begin{multicols}{2}
	\شروع{فقرات}
	\فقره گردش بستانکار
	\فقره گردش یدهکار
	\فقره مجموع سود حساب‌ها
	\فقره مانده ابتدای سال
	\فقره مانده انتهای سال
	\پایان{فقرات}
	
\end{multicols}

\فقره \textbf{اطلاعات تراکنش} (سال ۹۸ و شش ماههٔ اول سال ۹۹ و به تفکیک سال)
\begin{multicols}{2}
	\شروع{فقرات}
	\فقره مقدار کل تراکنش کارت‌ها 
	\فقره تعداد کل تراکنش کارت‌ها 
	\پایان{فقرات}
	
\end{multicols}

\فقره \textbf{دارایی‌ها}
\begin{multicols}{2}
	\شروع{فقرات}
	\فقره تعداد خودروهای فرد
	\فقره مجموع ارزش خودروهای فرد 
	\پایان{فقرات}
	
\end{multicols}


\فقره \textbf{تفریحات} (در بازهٔ سال‌های ۹۶ تا ۹۹)
\begin{multicols}{2}
	\شروع{فقرات}
	\فقره تعداد سفرهای خارجی هوایی غیرزیارتی
	\فقره تعداد سفرهای خارجی زمینی غیرزیارتی
	\فقره تعداد سفرهای خارجی هوایی زیارتی
	\فقره تعداد سفرهای خارجی زمینی زیارتی
	\پایان{فقرات}
	
\end{multicols}


\فقره \textbf{کسب و کار}
\begin{multicols}{2}
	\شروع{فقرات}
	\فقره داشتن مجوز صنفی
	\فقره صنفی که در آن مجوز دارد 
	\پایان{فقرات}
	
\end{multicols}


\فقره \textbf{بیمه}
\begin{multicols}{2}
	\شروع{فقرات}
	\فقره داشتن بیمه سلامت
	\فقره نوع بیمه سلامت
	\پایان{فقرات}
	
\end{multicols}


\فقره \textbf{شرایط خاص}
\begin{multicols}{2}
	\شروع{فقرات}
	\فقره داشتن بیماری خاص
	\فقره معلول بودن فرد
	\پایان{فقرات}
	
\end{multicols}


\فقره \textbf{بازنشستگی}
\begin{multicols}{2}
	\شروع{فقرات}
	\فقره بیمه‌پرداز صندوق‌های بازنشستگی بودن
	\فقره بازنشسته صندوق‌های بازنشستگی بودن
	\پایان{فقرات}
	
\end{multicols}


\فقره \textbf{مالیات و درآمد}
\begin{multicols}{2}
	\شروع{فقرات}
	\فقره شاغل مشمول مالیات بودن
	\فقره مجموع درآمد فرد از حقوق
	\پایان{فقرات}
	
\end{multicols}

\پایان{فقرات}



\قسمت{آماره‌ها و شهود}


\زیرقسمت{اندازه خانوار}

\شروع{شکل}[hb]
\centerimg{EDA/FamilySizeHistogramLog}{10cm}
\شرح{نمودار هیستوگرام اندازهٔ خانوارهای موجود در نمونه ۲ درصدی پایگاه اطلاعات رفاه ایرانیان. در نمودار بالا محور عمودی به صورت لگاریتمی مقیاس شده‌است.}
\برچسب{شکل:لگاریتم هیستوگرام سایز خانوار}
\پایان{شکل}



\زیرقسمت{توزیع خانوارها در کشور}

\شروع{شکل}[hb]
\centerimg{EDA/ProvinceFamilyCount}{15cm}
\شرح{نمودار تعداد خانوارهای موجود در نمونه ۲ درصدی پایگاه اطلاعات رفاه ایرانیان به تفکیک استان محل زندگی. بیشترین تعداد خانوار مربوط به استان تهران و کمترین آن مربوط به استان ایلام می‌باشد.}
\پایان{شکل}


\زیرقسمت{}


